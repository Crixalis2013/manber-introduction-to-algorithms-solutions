%%% File: 3.5.tex
%%% Author: Alexandre Medeiros <alexandre.medeiros@students.ic.unicamp.br>

\paragraph{3.5}

\subparagraph{a.}

We have that $f(n) = 100n + \log n$ and $g(n) = n + (\log n)^2$.
Then let
\[
  \lim_{n \to \infty} \frac{f(n)}{g(n)} =
  \lim_{n \to \infty} \frac{100n + \log n}{n + (\log n)^2}
\]
By L'Hôpital's rule
\[
  \lim_{n \to \infty} \frac{100n + \log n}{n + (\log n)^2} =
  \lim_{n \to \infty} \frac{100 + \frac{1}{n}}{1 + 2\frac{\log n}{n}} = 100
\]
\begin{equation}\label{eq:ch03:5:a:O}
  \therefore f(n) = O(g(n))
\end{equation}
Now, let
\[
  \lim_{n \to \infty} \frac{g(n)}{f(n)} =
  \lim_{n \to \infty} \frac{n + (\log n)^2}{100n + \log n}
\]
By L'Hôpital's rule
\[
  \lim_{n \to \infty} \frac{n + (\log n)^2}{100n + \log n} =
  \lim_{n \to \infty} \frac{1 + 2\frac{\log n}{n}}{100 + \frac{1}{n}} =
  \frac{1}{100}
\]
\begin{equation}\label{eq:ch03:5:a:Om}
  \therefore g(n) = O(f(n)) \Leftrightarrow f(n) = \Omega(g(n))
\end{equation}
From \eqref{eq:ch03:5:a:O} and \eqref{eq:ch03:5:a:Om} we have that
\[
  f(n) = \Theta(g(n))
\]

\subparagraph{b.}

We have that $f(n) = \log n$ and $g(n) = \log n^2$.
Note that
\[
  g(n) = \log n^2 = 2\log n = 2 f(n)
\]
\[
  \therefore f(n) = \Theta(g(n))
\]

\subparagraph{c.}

We have that $f(n) = \frac{n^2}{\log n}$ and $g(n) = n(\log n)^2$.
Then let
\[
  \lim_{n \to \infty} \frac{f(n)}{g(n)} =
  \lim_{n \to \infty} \frac{\frac{n^2}{\log n}}{n(\log n)^2} =
  \lim_{n \to \infty} \frac{n}{(\log n)^3}
\]
By L'Hôpital's rule
\[
  \lim_{n \to \infty} \frac{n}{(\log n)^3} =
  \lim_{n \to \infty} \frac{n}{3(\log n)^2} =
  \lim_{n \to \infty} \frac{n}{6\log n} =
  \lim_{n \to \infty} \frac{n}{6} = \infty
\]
\[
  \therefore f(n) \neq O(g(n))
\]
Now, let
\[
  \lim_{n \to \infty} \frac{g(n)}{f(n)} =
  \lim_{n \to \infty} \frac{n(\log n)^2}{\frac{n^2}{\log n}} =
  \lim_{n \to \infty} \frac{(\log n)^3}{n}
\]
By L'Hôpital's rule
\[
  \lim_{n \to \infty} \frac{(\log n)^3}{n} =
  \lim_{n \to \infty} \frac{3(\log n)^2}{n} =
  \lim_{n \to \infty} \frac{6\log n}{n} =
  \lim_{n \to \infty} \frac{6}{n} = 0
\]
\[
  \therefore g(n) = O(f(n)) \Leftrightarrow f(n) = \Omega(g(n))
\]
